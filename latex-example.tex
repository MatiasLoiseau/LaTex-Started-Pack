

% This is a semi-simple sample document.

\documentclass{article} % \documentclass{} is the first command in any LaTeX code.  It is used to define what kind of document you are creating such as an article or a book, and begins the document preamble

%\usepackage[spanish]{babel} % Para traducir palabras claves al español como "Figure" -> "Figura"

\usepackage{amsmath, amssymb} % \usepackage is a command that allows you to add functionality to your LaTeX code

\usepackage{mathtools} % using cases inside equations

\usepackage{graphicx} % image package

\usepackage{float} % Float specifiers are written in the square brackets whenever we use a float such as a figure or a table

\usepackage{subcaption} % more than one picture s in the same figure

\parskip 0.1in %paragraph distance

\usepackage[margin=0.984252in]{geometry} %margin

\usepackage[hidelinks]{hyperref} % Magic for index linking

\usepackage{chngcntr} % For figure number matching with section number
\counterwithin{figure}{section}

\usepackage{setspace} % Set space between paragraphs

\usepackage{appendix} % Appendix

%\pagestyle{headings} % Headings -> put the name of the section and page number

%\pagestyle{myheadings} % Personalized headings
%\markright{whateveryouwant}

\title{Semi-Simple Sample Document} % Sets article title
\date{November 2021} % Sets article date

\author{
	\textbf{Ingeniería en Informática}\\
	Departamento de Tecnología y Administración\\
	\\~\\
	\textbf{Loiseau, Matías}\\
	mloiseau@undav.edu.ar
 	\\~\\
 	\textbf{Snow, Jon}\\
 	jsnow@winterfell.edu.ar
}

% The preamble ends with the command \begin{document}
\begin{document} % All begin commands must be paired with an end command somewhere

\begin{figure}
\centering
	\includegraphics[width=0.2\textwidth]{images/undav-logo}
	%\caption{Mi Figure}
	\label{fig:undav-logo}
\end{figure}
\maketitle % creates title using information in preamble (title, author, date)

\thispagestyle{empty} % Ignore page number
\cleardoublepage

\cleardoublepage
\tableofcontents % general index
\cleardoublepage

\section{Forms to write text} % creates a section
Normal text

\textbf{Text in bold (ctrl + b)}

\textit{Text in italic (ctrl + i)}

Text with footnote\footnote{This is a footnote.}

This is a cite\cite{knn}.

\subsection{Enumerate and itemize information}

\begin{enumerate}
	\item Enumerate one
	\item Enumerate two
	\begin{enumerate}
		\item Sub-enumerate one
		\item Sub-enumerate two
	\end{enumerate}
\end{enumerate}

\begin{itemize}
	\item item one
	\item item two
	\begin{itemize}
		\item sub-item one
		\item sub-	item two
	\end{itemize}
\end{itemize}

\begin{enumerate}\addtocounter{enumi}{-1}
	  \item Start enumerate at 0
	  \item Enumerate one
\end{enumerate}

\section{Spaces}

Add space between \quad words.

Carbon monoxide (CO) is a colorless, odorless, and tasteless gas composed of one carbon atom and one oxygen atom. It is produced through incomplete combustion of carbon-containing fuels such as gasoline, natural gas, coal, wood, and oil. When these fuels do not burn completely due to insufficient oxygen supply, carbon monoxide is formed instead of carbon dioxide (CO2).

\begin{spacing}{0.8} 
Carbon monoxide (CO) is a colorless, odorless, and tasteless gas composed of one carbon atom and one oxygen atom. It is produced through incomplete combustion of carbon-containing fuels such as gasoline, natural gas, coal, wood, and oil. When these fuels do not burn completely due to insufficient oxygen supply, carbon monoxide is formed instead of carbon dioxide (CO2).
\end{spacing}

\begin{spacing}{1.5} 
Carbon monoxide (CO) is a colorless, odorless, and tasteless gas composed of one carbon atom and one oxygen atom. It is produced through incomplete combustion of carbon-containing fuels such as gasoline, natural gas, coal, wood, and oil. When these fuels do not burn completely due to insufficient oxygen supply, carbon monoxide is formed instead of carbon dioxide (CO2).
\end{spacing}

% util function: 
%$\begin{spacing}{0.9} \tableofcontents \end{spacing}

\section{Equations}

\subsection{Simple equations}

\begin{equation}
	E=mc^2
\end{equation}

\begin{equation}
	f(x) =
	\begin{cases*}
		1 & if x $>$ 0\\
 		0 & if x $\leqslant$ 0
 	 \end{cases*}
\end{equation}

\subsection{Complex equations}

\begin{equation}\label{eq:layers}
	\vec{h_1}^{\,} = f(\vec{x}^{\,}.W_1)
\end{equation}
	$$\vec{h_2}^{\,} = f(\vec{h_1}^{\,}.W_2)$$
	$$\vec{h_3}^{\,} = f(\vec{h_2}^{\,}.W_3)$$
	$$\vec{y}^{\,} = f(\vec{h_3}^{\,}.W_4)$$

\begin{equation}\label{eq:mse}
	MSE=\frac{1}{n}\sum_{i=1}^{n}(y_{i}-\hat{y}_{i})^2
\end{equation}

$$
\begin{pmatrix} a_0 & a_1\\ a_2 & a_3 \end{pmatrix}
\odot
\begin{pmatrix} b_0 & b_1\\ b_2 & b_3 \end{pmatrix}
=
\begin{pmatrix} a_0.b_0 & a_1.b_1\\ a_2.b_2 & a_3.b_3 \end{pmatrix}
$$

$$\frac{dL}{db_k}=\frac{dL}{dy_k}\frac{dy_k}{db_k}=\frac{dL}{dy_k}\frac{dy_k}{dz_k}\frac{dz_k}{db_k}=l'_{k+1}\odot f'_k\frac{d(W_kx_k+b_k)}{db_k}$$

\subsubsection{Matrix}

\begin{equation}
IoU(A,B)=\frac{|A \cap B|}{|A \cup B|}=\frac{|A \cap B|}{|A| + |B| - |A \cap B|}
\end{equation}

\cleardoublepage

\section{Figures} % creates a section

\begin{figure}[H]
	\centering
	\includegraphics[width=0.8\textwidth]{images/perceptron-graph}
	\caption{Perceptron diagram.}
	\label{fig:perceptron-graph}
\end{figure}

This is a reference for the image (figure \ref{fig:perceptron-graph}) above.

\begin{figure}[H]
	\centering
	\begin{subfigure}{2in}
		\includegraphics[width=1\textwidth]{images/mse-visu-1}
		\caption{Figure one.}
		\label{fig:mse-visu-1}
	\end{subfigure}	
	\begin{subfigure}{2in}
		\includegraphics[width=1\textwidth]{images/mse-visu-2}
		\caption{Figure two.}
		\label{fig:mse-visu-2}
	\end{subfigure}
	\caption{Both figures.}
\label{fig:mse-visu-12}
\end{figure}

\section{Tables}

\begin{table}[H]
	\centering
		\begin{tabular}{||l | c ||}
			\hline
			\hline
			Data 1 & 185\\
			\hline
			Data 222 & 37\\
			\hline
			Data 333333 & 12\\
			\hline
			Data 4444444444 & 12\\
			\hline
			Data 555 & 13\\
			\hline
			\hline
		\end{tabular}
		\caption{Table X.}
	\label{tab:table-x}
\end{table}

\begin{table}[H]
	\centering
		\begin{tabular}{||l || c | c | c | c | c ||}
			\hline
			\hline
			& \multicolumn{3}{c|}{Number of people} & \multicolumn{2}{c||}{Difference between}\\
			\hline
			Module & 0 & 1 & 2 & 1 y 0 & 2 y 0\\
			\hline			
			\hline
			1 & 21.07 & 21.19 & 21.33 & 0.12 & 0.26\\
			\hline
			2 & 22.25 & 22.35 & 22.39 & 0.1 & 0.14\\
			\hline
			3 & 18.11 & 18.19 & 18.61 & 0.07 & 0.5\\
			\hline
			4 & 18.3 & 18.44 & 18.87 & 0.15 & 0.57\\
			\hline
			6 & 18.21 & 18.43 & 19.45 & 0.21 & 1.24\\
			\hline
			\hline
		\end{tabular}
		\caption{Table Y.}
	\label{tab:table-y}
\end{table}


\cleardoublepage

\appendix
\clearpage
\addappheadtotoc
\appendixpage

\section{First Appendix}\label{app:one}

This is an appendix

\cleardoublepage

\section{Second Appendix}\label{app:two}

Text

\cleardoublepage

\begin{thebibliography}{9}
\addcontentsline{toc}{section}{Bibliography} % This is for the bibliography appears in the index

\bibitem{iot}
Iván Federico Kwist, Matías Loiseau, David Exequiel Contreras, Federico Gabriel D’Angiolo,  Roberto Osvaldo Mayer. (2019). \textit{Monitorización de un Datacenter mediante Protocolos de IoT}. Congreso Nacional de Ingeniería Informática – Sistemas de Información.

\bibitem{regresion-lineal}
Federico Gabriel D’Angiolo, Iván Federico Kwist, Matías Loiseau, David Exequiel Contreras, Fernando Asteasuain. (2019). \textit{Algoritmos de Regresión Lineal aplicados al mantenimiento de un Datacenter}. Congreso Argentino de Ciencias de la Computación.

\bibitem{knn}
Federico Gabriel D’Angiolo, Iván Federico Kwist, Matías Loiseau , David Exequiel Contreras, Gregorio Oscar Glas. (2019). \textit{Algoritmo de KNN aplicado al mantenimiento de un Datacenter}. Congreso Nacional de Ingeniería Informática – Sistemas de Información.

\bibitem{deep-learning-nature}
LeCun, Y., Bengio, Y., \& Hinton, G. (2015). \textit{Deep learning}. nature, 521(7553), 436-444.

\bibitem{object-detection-review}
Zhao, Z. Q., Zheng, P., Xu, S. T., \& Wu, X. (2019). \textit{Object detection with deep learning: A review}. IEEE transactions on neural networks and learning systems, 30(11), 3212-3232.

\end{thebibliography}


\end{document} % This is the end of the document
